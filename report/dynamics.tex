\subsection{Динамика манипулятора}\label{part_dynamics}

\subsubsection{Общие замечания}
Введем в рассмотрение барицентрические СК $Ox_{ci}y_{ci}z_{ci}$\lefteqn,\footnote{Системы координат, чьи начала совпадают с центрами масс соответствующих звеньев.} где $i=\overline{1,5}$, показанные на рисунке~\ref{img_mass_frames}.
Заметим, что каждая СК $Ox_{ci}y_{ci}z_{ci}$ сонаправлена с~$Ox_iy_iz_i$.

\begin{figure}[h!]
	\centering\includegraphics[height=16.5cm]{kinematics_mass_frames.pdf}
	\caption{Положение барицентрических СК и направление вектора $\vec{g}$.}
	\label{img_mass_frames}
\end{figure}

Для описания положения введенных СК воспользуемся следующими векторами:
\begin{equation}
    r^i_{i,\,ci} =
    \begin{bmatrix}
        x_{ci} \\ y_{ci} \\ z_{ci}
    \end{bmatrix}\!\!,\quad i = \overline{1,5},
\end{equation}
где $x_{ci}$, $y_{ci}$ и $z_{ci}$~--- некоторые постоянные величины.

Для компонент тензоров инерции $\mathcal{I}^{i}_i = const$ введем следующие обозначения:
\begin{equation}
    \mathcal{I}^{i}_i =
    \begin{bmatrix}
        I_{i,\,xx} & I_{i,\,xy} & I_{i,\,xz} \\
        I_{i,\,xy} & I_{i,\,yy} & I_{i,\,yz} \\
        I_{i,\,xz} & I_{i,\,yz} & I_{i,\,zz}
    \end{bmatrix}\!\!\ldotp
\end{equation}

Заметим, что
\begin{equation}
    g_0 =
    \begin{bmatrix}
        0 \\ 0 \\ -g
    \end{bmatrix}\!\!,
\end{equation}
где $g=9.82\text{ м}/\text{с}^2$.

В~заключении раздела приведем формулы для расчета величин, которые потребуются в дальнейшем (везде $i = \overline{1,5}$):
\begin{itemize}
    \item для расчета $r^0_{0,\,i}$ и ${}^{0}R_i$ (см.~Приложение~\ref{app_ht_matrices}):
        \begin{equation}
            {}^0A_i = {}^0A_1 \cdot {}^1A_2 \cdot \ldots \cdot {}^{i-1}A_i;
        \end{equation}
    \item для расчета $r^i_{0,\,i}$:
        \begin{gather}
            r^i_{0,\,i} = {}^{0}R_i^T \cdot r^0_{0,\,i};
        \end{gather}
    \item для расчета $z^0_i$:
        \begin{equation}
            z^0_i = {}^{0}R_i \cdot z^i_i = {}^{0}R_i \cdot
            \begin{bmatrix}
                0 \\ 0 \\ 1
            \end{bmatrix}\!\!;
        \end{equation}
    \item для расчета $g_i$, $v^i_i$ и $\omega^i_i$:
        \begin{equation}
            g_i = {}^{0}R_i^T \cdot g_0,
            \qquad
            v^i_i = {}^{0}R_i^T \cdot v^0_i,
            \qquad
            \omega^i_i = {}^{0}R_i^T \cdot \omega^0_i \ldotp
        \end{equation}
\end{itemize}

\subsubsection{Вывод уравнений движения}
Потенциальная энергия манипулятора
\begin{equation}
    U =  -\sum_{i=1}^5 \left( m_i g_i^T r^i_{0,\,ci} \right) = -\sum_{i=1}^5 \left( m_i g_i^T r^i_{0,\,i} + g_i^T (m_ir^i_{i,\,ci}) \right)\!,
\end{equation}


Якобианы, устанавливающие в соответствии с формулой
\begin{equation}
    v^0_{i} = J_{vi}\dot{q}, \quad i = \overline{1,5}
\end{equation}
связь между линейными скоростями начал соответствующих СК и вектором~$\dot{q}$:
\begin{gather}
    J_{v1} =
    \begin{bmatrix}
        z^0_0 \times \left( r^0_{0,\,1} - r^0_{0,\,0}\right) & \nv & \nv & \nv & \nv
    \end{bmatrix}\!\!,
    \\
    J_{v2} =
    \begin{bmatrix}
        z^0_0 \times \left( r^0_{0,\,2} - r^0_{0,\,0}\right) & z^0_1 \times \left( r^0_{0,\,2} - r^0_{0,\,1}\right) & \nv & \nv & \nv
    \end{bmatrix}\!\!,
    \\
    J_{v3} =
    \begin{bmatrix}
        z^0_0 \times \left( r^0_{0,\,3} - r^0_{0,\,0}\right) & z^0_1 \times \left( r^0_{0,\,3} - r^0_{0,\,1}\right) &
        z^0_2 \times \left( r^0_{0,\,3} - r^0_{0,\,2}\right) & \nv & \nv
    \end{bmatrix}\!\!,
    \\
    J_{v4} =
    \begin{bmatrix}
        z^0_0 \times \left( r^0_{0,\,4} - r^0_{0,\,0}\right) \\
        z^0_1 \times \left( r^0_{0,\,4} - r^0_{0,\,1}\right) \\
        z^0_2 \times \left( r^0_{0,\,4} - r^0_{0,\,2}\right) \\
        z^0_3 \times \left( r^0_{0,\,4} - r^0_{0,\,3}\right) \\
        \nv
    \end{bmatrix}^T\!\!\!\!\!,
    \qquad
    J_{v5} =
    \begin{bmatrix}
        z^0_0 \times \left( r^0_{0,\,5} - r^0_{0,\,0}\right) \\
        z^0_1 \times \left( r^0_{0,\,5} - r^0_{0,\,1}\right) \\
        z^0_2 \times \left( r^0_{0,\,5} - r^0_{0,\,2}\right) \\
        z^0_3 \times \left( r^0_{0,\,5} - r^0_{0,\,3}\right) \\
        z^0_4 \times \left( r^0_{0,\,5} - r^0_{0,\,4}\right)
    \end{bmatrix}^T\!\!\!\!\!,
\end{gather}
где $\nv = [0\;0\;0]^T$~--- нулевой вектор.

Якобианы, устанавливающие в соответствии с формулой
\begin{equation}
    \omega^0_{i} = J_{\omega i}\dot{q}, \quad i = \overline{1,5}
\end{equation}
связь между угловыми скоростями звеньев и вектором~$\dot{q}$:
\begin{gather}
    J_{\omega 1} =
    \begin{bmatrix}
        z^0_0 & \nv & \nv & \nv & \nv
    \end{bmatrix}\!\!,
    \qquad
    J_{\omega 2} =
    \begin{bmatrix}
        z^0_0 & z^0_1 & \nv & \nv & \nv
    \end{bmatrix}\!\!,
    \\
    J_{\omega 3} =
    \begin{bmatrix}
         z^0_0 & z^0_1 & z^0_2 & \nv & \nv
    \end{bmatrix}\!\!,
    \qquad
    J_{\omega 4} =
    \begin{bmatrix}
        z^0_0 & z^0_1 & z^0_2 & z^0_3 & \nv
    \end{bmatrix}\!\!,
    \\
    J_{\omega 5} =
    \begin{bmatrix}
        z^0_0 & z^0_1 & z^0_2 & z^0_3 & z^0_4
    \end{bmatrix}\!\!\ldotp
\end{gather}

Кинетическая энергия манипулятора
\begin{equation}
    K = \sum_{i=1}^5 \left( \frac{1}{2} m_i (v^i_i)^T v^i_i + \frac{1}{2} (\omega^i_i)^T \mathcal{I}^{i}_i \omega^i_i + (m_ir^i_{i,\,ci})^T \cdot (v^i_i \times \omega^i_i) \right)  \ldotp
\end{equation}

Функция Лагранжа
\begin{gather}
    L = K - U = \notag
    \\
    = \sum_{i=1}^5 \Biggl( m_i \left( \frac{1}{2} (v^i_i)^T v^i_i + g_i^T r^i_{0,\,i} \right) + (m_ir^i_{i,\,ci})^T \cdot \left( v^i_i \times \omega^i_i + g_i \right) + \frac{1}{2} (\omega^i_i)^T \mathcal{I}^{i}_i \omega^i_i \Biggr) = \notag
    \\
    = \sum_{i=1}^5 \Biggl( m_i \underbrace{\left( \frac{1}{2} (v^i_i)^T v^i_i + g_i^T r^i_{0,\,i} \right)}_{\ds L_{i,1}} + m_i x_{ci} \cdot \underbrace{x\left\{ v^i_i \times \omega^i_i + g_i \right\}}_{\ds L_{i,2}} + \notag
    \\
    + m_i y_{ci} \cdot \underbrace{y\left\{ v^i_i \times \omega^i_i + g_i \right\}}_{\ds L_{i,3}} + m_i z_{ci} \cdot \underbrace{z\left\{ v^i_i \times \omega^i_i + g_i \right\}}_{\ds L_{i,4}} + I_{i,\,xx} \cdot \underbrace{\frac{1}{2} \cdot \bigl(x\{\omega^i_i\}\bigr)^2}_{\ds L_{i,5}} +\notag
    \\
    + I_{i,\,yy} \cdot \underbrace{\frac{1}{2} \cdot \bigl(y\{\omega^i_i\}\bigr)^2}_{\ds L_{i,6}} + I_{i,\,zz} \cdot \underbrace{\frac{1}{2} \cdot \bigl(z\{\omega^i_i\}\bigr)^2}_{\ds L_{i,7}} + I_{i,\,xy} \cdot \underbrace{x\{\omega^i_i\} \cdot y\{\omega^i_i\}}_{\ds L_{i,8}} +\notag
    \\
    + I_{i,\,xz} \cdot \underbrace{x\{\omega^i_i\} \cdot z\{\omega^i_i\}}_{\ds L_{i,9}} + I_{i,\,yz} \cdot \underbrace{y\{\omega^i_i\} \cdot z\{\omega^i_i\}}_{\ds L_{i,10}}\Biggr) \ldotp
\end{gather}

Уравнения движения робота:
\begin{equation}
    \frac{d}{dt}\frac{\partial L}{\partial\dot{q_i}} - \frac{\partial L}{\partial q_i} = \tau_i, \quad i = \overline{1,5} \qquad \Rightarrow
\end{equation}
\begin{equation}
    \Rightarrow \quad
	\left\{
	\begin{aligned}
		\!&\sum_{i=1}^5 \bigl( m_i \cdot \mathcal{L}_1 \{L_{i,1}\} + m_i x_{ci} \cdot \mathcal{L}_1 \{L_{i,2}\} + \ldots + I_{i,\,yz} \cdot \mathcal{L}_1 \{L_{i,10}\} \bigr) = \tau_1\\
		\!&\sum_{i=1}^5 \bigl( m_i \cdot \mathcal{L}_2 \{L_{i,1}\} + m_i x_{ci} \cdot \mathcal{L}_2 \{L_{i,2}\} + \ldots + I_{i,\,yz} \cdot \mathcal{L}_2 \{L_{i,10}\} \bigr) = \tau_2\\
		\!&\ldots\\
		\!&\sum_{i=1}^5 \bigl( m_i \cdot \mathcal{L}_5 \{L_{i,1}\} + m_i x_{ci} \cdot \mathcal{L}_5 \{L_{i,2}\} + \ldots + I_{i,\,yz} \cdot \mathcal{L}_5 \{L_{i,10}\} \bigr) = \tau_5
	\end{aligned}
	\right.
\end{equation}
где $\mathcal{L}_j$~--- оператор, работающий в соответствии с формулой:
\begin{equation}
    \mathcal{L}_j : \quad \mathcal{L}_j \{f\} = \frac{d}{dt}\frac{\partial f}{\partial\dot{q_j}} - \frac{\partial f}{\partial q_j},
\end{equation}
где в свою очередь $f = f(\dot{q}(t), q(t))$.
Если же заметить, что
\begin{equation}
    \mathcal{L}_j \{L_{i,k}\} = 0 \qquad \text{при }j > i, \quad i,j=\overline{1,5}, \quad k=\overline{1,10},
\end{equation}
то выражения для них упрощаются до:
\begin{equation}
	\left\{
	\begin{aligned}
		\!&\sum_{i=1}^5 \bigl( m_i \cdot \mathcal{L}_1 \{L_{i,1}\} + m_i x_{ci} \cdot \mathcal{L}_1 \{L_{i,2}\} + \ldots + I_{i,\,yz} \cdot \mathcal{L}_1 \{L_{i,10}\} \bigr) = \tau_1\\
		\!&\sum_{i=2}^5 \bigl( m_i \cdot \mathcal{L}_2 \{L_{i,1}\} + m_i x_{ci} \cdot \mathcal{L}_2 \{L_{i,2}\} + \ldots + I_{i,\,yz} \cdot \mathcal{L}_2 \{L_{i,10}\} \bigr) = \tau_2\\
		\!&\ldots\\
		\!& m_5 \cdot \mathcal{L}_5 \{L_{5,1}\} + m_5 x_{c5} \cdot \mathcal{L}_5 \{L_{5,2}\} + \ldots + I_{5,\,yz} \cdot \mathcal{L}_5 \{L_{5,10}\} \bigr) = \tau_5
	\end{aligned}
	\right.
\end{equation}
или в матричном виде
\begin{equation}
    \tau = \xi \chi,
\end{equation}
где $\tau = [\tau_1, \: \tau_2, \: \ldots, \: \tau_5]^T$~--- вектор обобщенных моментов,\\ $\chi=[\chi_1, \: \chi_2, \: \ldots, \: \chi_5]^T \in \mathbb R^{50}$~--- вектор параметров робота, где в свою очередь
\begin{equation}
    \chi_i =
    \begin{bmatrix}
        m_i & m_i x_{ci} & m_i y_{ci} & m_i y_{ci} & I_{i,\,xx} & I_{i,\,yy} & I_{i,\,zz} & I_{i,\,xy} & I_{i,\,xz} & I_{i,\,yz}
    \end{bmatrix}^T\!\!\!\!;
\end{equation}
$\xi$~--- так называемый регрессор, равный
\begin{equation}
    \xi =
    \begin{bmatrix}
        \xi_{1,1} & \xi_{1,2} & \cdots & \xi_{1,5} \\
        O_{1 \times 10} & \xi_{2,2} & \cdots & \xi_{2,5} \\
        \vdots & \vdots & \ddots & \vdots \\
        O_{1 \times 10} & O_{1 \times 10} & O_{1 \times 10} & \xi_{5,5}
    \end{bmatrix}\!\!,
\end{equation}
где в свою очередь $O_{1 \times 10}$~--- вектор-строка, состоящая из 10 нулей, а $\xi_{j,i} =$\linebreak $= \xi_{j,i}(\ddot{q}, \dot{q}, q)$~--- вектор-строка, рассчитываемый по формуле
\begin{equation}
    \xi_{j,i} =
    \begin{bmatrix}
        \mathcal{L}_j \{L_{i,1}\} & \mathcal{L}_j \{L_{i,2}\} & \ldots & \mathcal{L}_j \{L_{i,10}\}
    \end{bmatrix}\!\!\ldotp
\end{equation}
\newpage

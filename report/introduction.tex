\section*{Введение}
\addcontentsline{toc}{section}{Введение}
В~данном документе будет рассказано о процессе разработки системы управления для манипулятора робота Kuka Youbot~\cite{technomatix_kuka_youbot}, дающей ему возможность для совершения двух действий: занятия позиции, при которой его схват будет принимать заданные положение и ориентацию, а также перемещения схвата по заданной траектории\footnote{Здесь и далее, когда речь будет идти о траектории движении схвата, под последней будет подразумеваться не просто кривая, описываемая при этом схватом в пространстве, но таковая, явно параметризованная временем.}.
В~целом содержание пояснительной записки можно описать примерно так:
\begin{itemize}
\item в~разделе~\ref{part_description_of_robot} будут приведены технические сведения о роботе, необходимые для решения поставленных задач;
\item раздел~\ref{part_math_model_of_robot} расскажет о процессе составления математической модели манипулятора, а именно о решении применительно к нему прямой и обратной задач кинематики и о составлении дифференциальных уравнений, описывающих протекающие в роботе электрические и механические процессы;
\item в~разделе~\ref{part_control_systems} речь пойдет о синтезе соответствующих систем управления, о проверке их работоспособности с помощью моделирования, о результатах аппробации на реальном роботе и проч.
\end{itemize}
\newpage

\section{Математическая модель манипулятора}\label{part_math_model_of_robot}

\subsection{Кинематика манипулятора}\label{part_kinematics}

\subsubsection{Общие замечания}

% TODO Тут обосновать зачем нужна кинематика манипулятора

Последовательная кинематическая цепь рассматриваемого манипулятора, включающая только вращательные КП V-класса (цилиндрические шарниры), изображена на рисунке~\ref{img:kinematics}a.

\begin{figure}[h!]
	\begin{minipage}[h]{0.5\linewidth}
		\centering{\includegraphics[width=0.95\linewidth]{kinematics_schema.pdf} \\ а)}
	\end{minipage}
	\hfill
	\begin{minipage}[h]{0.5\linewidth}
		\centering{\includegraphics[width=0.95\linewidth]{kinematics_frames.pdf} \\ б)}
	\end{minipage}
	\caption{Схемы рассматриваемого манипулятора: а~--- кинематическая при $q_i=0$, $i=\overline{1,5}$; б~--- расположения СК КП.}
	\label{img:kinematics}
\end{figure}

Для описания положений звеньев манипулятора друг относительно друга воспользуемся методом Денавита--Хартенберга, состоящим из трех данных шагов:
\begin{enumerate}
    \item <<привязка>> к каждому звену СК, чьи оси удовлетворяют следующим условиям:
    \begin{enumerate}
	    \item ось $z_{i-1}$ направлена вдоль оси $i$-ой КП;
        \item ось $x_i$ перпендикулярна оси $z_{i-1}$ и пересекает ее;
        \item ось $y_i$ дополняет оси $z_i$ и $x_i$ до правой декартовой СК.
    \end{enumerate}
    \item определение параметров ДХ:
    \begin{enumerate}
	    \item $a_i$~--- расстояния от $z_{i-1}$ до $z_i$ вдоль $x_i$;
	    \item $\alpha_i$~--- угла от $z_{i-1}$ до $z_i$ вокруг $x_i$;
        \item $d_i$~--- расстояния от $x_{i-1}$ до $x_i$ вдоль $z_{i-1}$;
        \item $\theta_i$~--- угла от $x_{i-1}$ до $x_i$ вокруг $z_{i-1}$.
    \end{enumerate}
    \item расчет однородных матриц преобразования\footnote{За пояснениями обратитесь к Приложению~\ref{app_ht_matrices}} в соответствии со следующими формулами:
    \begin{equation}\label{DH_matrix}
        {}^{i-1}A_{i} = R_{z, \theta_i} \cdot T_{z, d_i} \cdot T_{x, a_i} \cdot R_{x, \alpha_i}
    \end{equation}
    где $R_{z, \theta_i}$~--- матрица поворота вокруг оси $z$ на угол $\theta_i$, $T_{z, d_i}$~--- матрица смещения вдоль оси $z$ на расстояние $d$, $T_{x, a_i}$~---матрица смещения вдоль оси $x$ на расстояние $a_i$,  $R_{x, \alpha_i}$~--- матрица поворота вокруг оси $x$ на угол $\alpha_i$, равные
    \begin{gather}
        R_{z, \theta_i} =
        \begin{bmatrix}
            \cos\theta_i & -\sin\theta_i & 0 & 0\\
            \sin\theta_i &  \cos\theta_i & 0 & 0\\
            0 & 0 & 1 & 0\\
            0 & 0 & 0 & 1
        \end{bmatrix}\!\!,
        \quad
        T_{z, d_i} =
        \begin{bmatrix}
            1 & 0 & 0 & 0\\
            0 & 1 & 0 & 0\\
            0 & 0 & 1 & d_i\\
            0 & 0 & 0 & 1\\
        \end{bmatrix}\!\!,\\
        %
        T_{x, a_i} =
        \begin{bmatrix}
            1 & 0 & 0 & a_i\\
            0 & 1 & 0 & 0\\
            0 & 0 & 1 & 0\\
            0 & 0 & 0 & 1\\
        \end{bmatrix}\!\!,
        \quad
        R_{x, \alpha_i} =
        \begin{bmatrix}
            1 & 0 & 0 & 0\\
            0 & \cos\alpha_i & -\sin\alpha_i & 0\\
            0 & \sin\alpha_i &  \cos\alpha_i & 0\\
            0 & 0 & 0 & 1
        \end{bmatrix}\!\!;
    \end{gather}
    итого
    \begin{equation}
        {}^{i-1}A_i =
        \begin{bmatrix}
            \cos\theta_i & - \cos\alpha_i \sin\theta_i & \sin\alpha_i \sin\theta_i & a_{i} \cos\theta_i\\
            \sin\theta_i & \cos\alpha_i \cos\theta_i & - \sin\alpha_i \cos\theta_i & a_{i} \sin\theta_i\\
            0 & \sin\alpha_i & \cos\alpha_i & d_{i}\\
            0 & 0 & 0 & 1
        \end{bmatrix}
    \end{equation}
\end{enumerate}

Результаты выполнения для исследуемого манипулятора двух первых шагов представлены на рисунке~\ref{img:kinematics}б и в таблице~\ref{table_DH_params}, а третьего~--- в лице следующих выражений:
\begin{gather}
	{}^0A_1\! =\!\!
    \left[\begin{matrix}c_{\theta_1} & 0 & s_{\theta_1} & a_{1} c_{\theta_1}\\s_{\theta_1} & 0 & - c_{\theta_1} & a_{1} s_{\theta_1}\\0 & 1 & 0 & d_{1}\\0 & 0 & 0 & 1\end{matrix}\right]\!\!;
    %
	{}^1A_2\! =\!\!
	\left[\begin{matrix}c_{\theta_2} & - s_{\theta_2} & 0 & a_{2} c_{\theta_2}\\s_{\theta_2} & c_{\theta_2} & 0 & a_{2} s_{\theta_2}\\0 & 0 & 1 & 0\\0 & 0 & 0 & 1\end{matrix}\right]\!\!;
    %
	{}^2A_3\! =\!\!
	\left[\begin{matrix}c_{\theta_3} & - s_{\theta_3} & 0 & a_{3} c_{\theta_3}\\s_{\theta_3} & c_{\theta_3} & 0 & a_{3} s_{\theta_3}\\0 & 0 & 1 & 0\\0 & 0 & 0 & 1\end{matrix}\right]\!\!;\notag
	\\
	{}^3A_4 =
	 \left[\begin{matrix}c_{\theta_4} & 0 & s_{\theta_4} & 0\\s_{\theta_4} & 0 & - c_{\theta_4} & 0\\0 & 1 & 0 & 0\\0 & 0 & 0 & 1\end{matrix}\right]\!\!;
	\;
	{}^4A_5 =
	\left[\begin{matrix}c_{\theta_5} & - s_{\theta_5} & 0 & 0\\s_{\theta_5} & c_{\theta_5} & 0 & 0\\0 & 0 & 1 & d_{5}\\0 & 0 & 0 & 1\end{matrix}\right]\!\!\ldotp
\end{gather}

\begin{table}[h!]
	\caption{Параметры Денавита-Хартенберга}
	\begin{center}
		\begin{tabular}{|c|c|c|c|c|}
			\hline
			Звено 	& $a_i$, мм & $\alpha_i$, рад & $d_i$, мм & $\theta_i$, рад\\
			\hline
			1  		& $33$ & $\pi/2$ & $147$ & $q_1$\\
			\hline	
			2 		& $155$ & $0$ 	& $0$ 	& $q_2 + \pi/2$\\
			\hline
			3 		& $135$ & $0$ 	& $0$ 	& $q_3$\\
			\hline
			4 		& $0$ & $\pi/2$ & $0$ 	& $q_4$\\
			\hline
			5 		& $0$ & $0$ 	& $218$ & $q_5$\\
			\hline
		\end{tabular}
	\end{center}
	\label{table_DH_params}
\end{table}


\subsubsection{Прямая задача кинематики}\label{part_kinematics_forward}

\if 0
% удалить
Представим прямую задачу кинематики функцией $f$, которая определяет зависимость между  конфигурационным и рабочим пространствами манипулятора:
\begin{equation}
	x = f(q)
\end{equation}
где $x \in \mathbb{R}^6$ --- вектор положения и ориентации схвата в рабочем пространстве, $q \in \mathbb{R}^5$ --- вектор обобщенных координат в конфигурационном пространстве манипулятора.
\fi

Представим прямую задачу кинематики (ПЗК) манипулятора выражением:
\begin{equation}\label{fk}
	^0A_6 = \prod^{6}_{i=1}{^{i-1}A_i(q_i)} = ^0A_1 \cdot ^1A_2 \cdot ^2A_3 \cdot ^3A_4 \cdot ^4A_5 \cdot ^5A_6
\end{equation}
где $^0A_6$~--- матрица $4 \times 4$, первые $3$ столбца которой представляют ориентацию, последний --- положение схвата; $^{i-1}A_i$~--- однородная матрица преобразования из $(i-1)$ в $i$-ую СК в общем виде:
\begin{equation}
	^{i-1}A_i = 
	\left[\begin{matrix}
	\cos{\left (\theta_i \right )} & - \sin{\left (\theta_i \right )} \cos{\left (\alpha_i \right )} & \sin{\left (\alpha_i \right )} \sin{\left (\theta_i \right )} & a_{i} \cos{\left (\theta_i \right )}\\
	\sin{\left (\theta_i \right )} & \cos{\left (\alpha_i \right )} \cos{\left (\theta_i \right )} & - \sin{\left (\alpha_i \right )} \cos{\left (\theta_i \right )} & a_{i} \sin{\left (\theta_i \right )}\\
	0 & \sin{\left (\alpha_i \right )} & \cos{\left (\alpha_i \right )} & d_{i}\\
	0 & 0 & 0 & 1
	\end{matrix}\right]
\end{equation}

Теперь, учитывая ДХ-параметры из таблицы~\ref{DH_matrix} находим матрцы преобразования СК, рисунок~\ref{img:kinematics} б.

\begin{align*}
	^0A_1 &=& 
	\left[\begin{matrix}1 & 0 & 0 & 0\\0 & 1 & 0 & 0\\0 & 0 & 1 & d_{1}\\0 & 0 & 0 & 1\end{matrix}\right]; 
	^1A_2 =
 	\left[\begin{matrix}c_{\theta_1} & 0 & s_{\theta_1} & a_{2} c_{\theta_1}\\s_{\theta_1} & 0 & - c_{\theta_1} & a_{2} s_{\theta_1}\\0 & 1 & 0 & 0\\0 & 0 & 0 & 1\end{matrix}\right];
	^2A_3 =
	\left[\begin{matrix}c_{\theta_2} & - s_{\theta_2} & 0 & a_{3} c_{\theta_2}\\s_{\theta_2} & c_{\theta_2} & 0 & a_{3} s_{\theta_2}\\0 & 0 & 1 & d_{2}\\0 & 0 & 0 & 1\end{matrix}\right];
	\\
	^3A_4 &=& 
	\left[\begin{matrix}c_{\theta_3} & - s_{\theta_3} & 0 & a_{4} c_{\theta_3}\\s_{\theta_3} & c_{\theta_3} & 0 & a_{4} s_{\theta_3}\\0 & 0 & 1 & 0\\0 & 0 & 0 & 1\end{matrix}\right];
	^4A_5 =
	 \left[\begin{matrix}c_{\theta_4} & 0 & s_{\theta_4} & 0\\s_{\theta_4} & 0 & - c_{\theta_4} & 0\\0 & 1 & 0 & 0\\0 & 0 & 0 & 1\end{matrix}\right];
	^5A_6 =
	\left[\begin{matrix}c_{\theta_5} & - s_{\theta_5} & 0 & 0\\s_{\theta_5} & c_{\theta_5} & 0 & 0\\0 & 0 & 1 & d_{6}\\0 & 0 & 0 & 1\end{matrix}\right]
\end{align*}

Таким образом, для любого вектора $q$, позьзуясь выражением~\eqref{fk} и ДХ-параметрами маниплятора, можно определить однозначное положение и ориентацию схвата манипулятора в пространстве.

Для проверки, зададим вектор обобщенных координат:
\begin{equation}
	q =
	\begin{bmatrix}
	\theta_1 & \theta_2 & \theta_3 & \theta_4 & \theta_5
	\end{bmatrix}
	=
	\begin{bmatrix}
	0 & 0 & 0 & 90 & 0
	\end{bmatrix}
\end{equation}

\begin{figure}[h]
	\centering
	\includegraphics[width=0.2\textwidth]{kinematics_check.pdf}
	\caption{Конфигурация манипулятора для заданного вектора $q$}
	\label{kinematics_check}
\end{figure}

В результате решения ПЗК должны получить:
\begin{align*}
	p = 
	\begin{bmatrix}
		0.033\\
		0\\
		0.655 
	\end{bmatrix},
	o =
	\begin{bmatrix}
		0\\
		0\\
		180
	\end{bmatrix}, 
\end{align*}
где $p$~--- положение схвата, $o$~--- ориентация схвата (крен, рыскание, тангаж).

Вычислим матрицу $^0A_6$:
\begin{equation}
	^0A_6 = 
	\left[\begin{matrix}-1 & 0 & 0 & 0.033\\0 & -1 & 0 & 0\\0 & 0 & 1 & 0.655\\0 & 0 & 0 & 1\end{matrix}\right]
\end{equation}

Из приведенного примера следует, что ДХ-параметры и матрицы трансформации найдены верно.

\subsubsection{Обратная задача кинематики}\label{part_kinematics_inverse}

% удалтить
Обратную задачу кинематики представим, как функцию $g = f^{-1}$, представляющую переход из рабочего в конфигурационное пространство:
\begin{equation}
\textbf{q} = g(\textbf{p}, \textbf{o}) = f^{-1}(\textbf{p}, \textbf{o})
\end{equation}
где вектор $\textbf{p}$~--- заданное положение в рабочем пространстве, вектор $\textbf{o}$~--- заданная ориентация системы координат схвата.

Для удобства будем пользоваться однородными матрицами преобразования.
Матрица, задающая положение и ориентацию схвата в системе координат базы, имеет вид:
\begin{equation}\label{ik}
	^0T_6 =
	\left[\begin{matrix}
	r_{11} & r_{12} & r_{13} & p^{x}\\
	r_{21} & r_{22} & r_{23} & p^{y}\\
	r_{31} & r_{32} & r_{33} & p^{z}\\
	0 & 0 & 0 & 1
	\end{matrix}\right]
\end{equation}

Приравняв матрицу $^0T_6$ и правую часть выражения~\eqref{fk} и домножив с обеих сторон на $(^0A_1 \cdot ^1A_2)^{-1}$, получим выражение:
\begin{equation}
	(^0A_1 \cdot ^1A_2)^{-1} \cdot ^0T_6 = ^2A_3 \cdot ^3A_4 \cdot ^4A_5 \cdot ^5A_6
\end{equation}
где левая часть:
\begin{align*}
	^2T_6 &= 
	\left[\begin{matrix}
		r_{11} c_{1} + r_{21} s_{1} & r_{12} c_{1} + r_{22} s_{1} & r_{13} c_{1} + r_{23} s_{1} & - a_{2} + p^{x} c_{1} + p^{y} s_{1}\\
		r_{31} & r_{32} & r_{33} & - d_{1} - d_{2} + p^{z}\\
		r_{11} s_{1} - r_{21} c_{1} & r_{12} s_{1} - r_{22} c_{1} & r_{13} s_{1} - r_{23} c_{1} & p^{x} s_{1} - p^{y} c_{1}\\
		0 & 0 & 0 & 1\end{matrix}\right],
\end{align*}
правая часть:
\begin{align*}
	^2A_6 &=
	\left[\begin{matrix}
		c_{5} c_{234} & - s_{5} c_{234} & s_{234} & a_{3} c_{2} + a_{4} c_{23} + d_{6} s_{234}\\
		s_{234} c_{5} & - s_{5} s_{234} & - c_{234} & a_{3} s_{2} + a_{4} s_{23} - d_{6} c_{234}\\
		s_{5} & c_{5} & 0 & 0\\
		0 & 0 & 0 & 1
	\end{matrix}\right].
\end{align*}

Теперь, приравнивая элементы с одинаковыми индексами получим уравнения, из которых найдем обобщенные координаты.

Из равенства элементов $(3,4)$: 
\begin{equation}
	p^{x} s_{1} - p^{y} c_{1} = 0
\end{equation}

Найдем $\theta_1$:
\begin{equation}
	\theta_1 = Atan2(p^y, p^x)
\end{equation}

Из равенств элементов $(3,1)$ и $(3,2)$:
\begin{align*}
	s_{5} = r_{11} s_{1} - r_{21} c_{1},\\
	c_{5} =  r_{12} s_{1} - r_{22} c_{1}
\end{align*}

Вычислим $\theta_5$:
\begin{equation}
	\theta_5 = Atan2(r_{11} s_{1} - r_{21} c_{1}, r_{12} s_{1} - r_{22} c_{1})
\end{equation}

Из равенств элементов $(2,3)$ и $(1,3)$:
\begin{align*}
	c_{234} &= -r_{33},\\
	s_{234} &= r_{13} c_{1} + r_{23} s_{1}
\end{align*}

Вычислим $\theta_{234}$:
\begin{equation}
	\theta_{234} = Atan2(r_{13} c_{1} + r_{23} s_{1}, -r_{33}) 
\end{equation}

Далее применим геометрический подход.

\begin{figure}[h!]
	\centering
	\includegraphics[width=0.7\textwidth]{ik_geometric_approach.pdf}
	\caption{Плоская часть манипулятора}
	\label{ik_geometric}
\end{figure}

Выпишем, пользуясь теоремой косинусов, выражения для $\theta_3$:
\begin{equation}
	\cos{\theta_3} = \frac{(^2p_4^x)^2 + (^2p_4^y)^2 + (^2p_4^z)^2 - a_3^2 - a_3^2}{2 a_3 a_4}
\end{equation}

\begin{equation}
	\theta_3^{1,2} = \mp Atan2(\sqrt{1 - \cos^2{\theta_3}}, \cos{\theta_3})
\end{equation}

Из рисунка~\ref{ik_geometric} видно, что, при $\theta_3 < 0$, $\theta_2 = \phi + \beta$:
\begin{equation}
	\theta_2^1 = Atan2(\sqrt{(^2p_4^x)^2 + (^2p_4^y)^2}, ^2p_4^z) + Atan2(a_4 \sin{\theta_3^1}, a_3 + a_4 \cos{\theta_3^1})
\end{equation}

При $\theta_3 > 0$, $\theta_2 = \phi - \beta$:
\begin{equation}
	\theta_2^2 = Atan2(\sqrt{(^2p_4^x)^2 + (^2p_4^y)^2}, ^2p_4^z) - Atan2(a_4 \sin{\theta_3^2}, a_3 + a_4 \cos{\theta_3^2})
\end{equation}

И, наконец:
\begin{equation}
	\theta_4^{1,2} = \theta_{234} - \theta_{2}^{1,2} - \theta_{3}^{1,2}
\end{equation}



\newpage